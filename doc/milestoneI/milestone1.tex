\documentclass[onecolumn]{aastex62}


\newcommand{\vdag}{(v)^\dagger}
\newcommand\aastex{AAS\TeX}
\newcommand\latex{La\TeX}
\usepackage{amsmath}
\usepackage{physics}
\usepackage{hyperref}
\usepackage{natbib}
\usepackage[T1]{fontenc}
\usepackage[english]{babel}
\usepackage[utf8]{inputenc}

\begin{document}

\title{\Large Milestone I:\\Computing the Hubble parameter and conformal time.}





\author{Håkon Tansem}



\section{Introduction} \label{sec:intro}
Milestone I is the first part of a project consisting of four Milestones. The
main goal of the project is to calculate the CMB power spectrum. In
Milestone I, we will study the background evolution of the universe. To do this
we will compute the Hubble parameter from early times to today as
to describe the expansion history of the universe. We will also study the
evolution of the density parameters for baryons, cold dark matter, photons and
cosmological constant as well as the conformal time.
 
\section{Method} \label{sec:method}
When calculating the conformal time we start with the differential equation
\begin{equation}
    \frac{d\eta}{dt} = \frac{c}{a},
\end{equation}
where $\eta$ is the conformal time, $c$ is the speed of light, $a$ is the scale
factor and $t$ is time. By performing a change of variables by multiplying with
$\frac{dx}{dx}$ we get
\begin{equation}
    \frac{d\eta}{dx}\frac{dx}{dt} = \frac{c}{a},
\end{equation}
where we introduce the coordinate change $x=log(a)$. Using that
$\frac{dx}{dt}=H$ (\cite{callin2006calculate}), where $H$ is the hubble paramater, we get an expression for $\eta$
\begin{equation}\label{eq:eta_of_x}
    \eta = \int_{0}^{x}\frac{c}{\mathcal{H}(x)}dx,
\end{equation}
where $\mathcal{H}(x)=e^xH(x)$ in our new coordinate system. The next step is to
derive an expression for $H(x)$. By neglecting density parameters for neutrinos
and curvature, we have and expression for the Hubble parameter given by
\begin{equation}\label{eq:H_of_x}
    H(x) = H_0\sqrt{(\Omega_{b,0} + \Omega_{CDM,0})e^{-3x} + \Omega_{r,0} e^{-4x} + \Omega_{\Lambda,0}}
\end{equation} 
(\cite{Winther:2020}), where $\Omega_{b,0}$ is the density parameter for baryons,
$\Omega_{CDM,0}$ is for cold dark matter, $\Omega_{r,0}$ is for radiation and
$\Omega_{\Lambda,0}$ is for cosmological constant. The sub index $0$ represents
the value for the parameters today. The values for the density parameters today
are given as follows
\begin{align}
    \Omega_{CDM,0} &=0.25\\
    \Omega_{b,0} &=0.05\\
    \Omega_{\Lambda,0} &=0.7\\
\end{align}
Respectively. The parameter for radiation today is given as 
\begin{equation}
    \Omega_{r,0} =2\frac{\pi^2(k_bT_{cmb}^4)}{30\hbar^3c^5}\frac{8\pi G}{3H_0^2},
\end{equation}
where $G$ is the gravitational constant, $H_0$ is the Hubble parameter today and
$T_{cmb} =2.7255$K is the temperature of the CMB measured today (\cite{Winther:2020}).
 

We also want to study how the density parameters evolve with time. For the density parameters, we
have that 
\begin{align}
    \Omega_i &= \frac{\rho_i}{\rho_{c}} \\
             &= \frac{8\pi G}{3H(x)^2}\rho_i\\
             &= \frac{8\pi G}{3H(x)^2}\rho_{i,0}e^{-3x(1+w_i)},
\end{align}
where $\rho_{i,0}$ is the critical density
today and $w$ is the equation of state parameter defined as $w_i\equiv\frac{P_i}{\rho_i}$, where $P_i$ is pressure and
$\rho_i$ is density for a given contribution to energy component $i$ (Spør om
dette). By using the relation $\rho_{i,0} = \rho_{c,0}\Omega_{i,0}$, we get the
expression for a given density parameter as 
\begin{equation}
    \Omega_i(x)=\frac{H_0^2}{H(x)^2}e^{-3x(1+w_i)}\Omega_{i,0}.
\end{equation}
Using that $w_{CDM} = 0$, $w_{b} = 0$, $w_{\Lambda} = -1$ and $w_{r} = 1/3$
(\cite{Winther:2020}) we can get calculate the density parameters for a given
$x$ allowing us to evaluate the Hubble parameter given by equation
(\ref{eq:H_of_x}) for a given x. This allows us to finally calculate the
integral for $\eta$ given by equation (\ref{eq:eta_of_x}).
\section{Results} \label{sec:results}


\bibliography{ref.bib}
\bibliographystyle{aasjournal}
%\begin{thebibliography}{}
%\end{thebibliography}
\end{document}

% End of file `sample62.tex'.
