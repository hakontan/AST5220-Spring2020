\documentclass[onecolumn]{aastex62}


\newcommand{\vdag}{(v)^\dagger}
\newcommand\aastex{AAS\TeX}
\newcommand\latex{La\TeX}
\usepackage{listings}
\usepackage{amsmath}
\usepackage{physics}
\usepackage{hyperref}
\usepackage{natbib}
\usepackage[T1]{fontenc}
\usepackage[english]{babel}
\usepackage[utf8]{inputenc}

\begin{document}

\title{\Large Milestone II:\\Studying the recombination history of the universe.}


\author{Håkon Tansem}

\section{Introduction} \label{sec:intro}
This is the second out of four milestones in a project where the final goal is to compute the CMB power spectrum. Previously we calculated the background evolution of the universe studying how the energy content and the particle horizon has evolved throughout cosmic history. Another important consideration to take into account when observing the CMB is to study the medium the photons have travelled from their release at the last scattering surface untill they reach us at present time. In this second milestone we will study how the optical depth of the universe changes as we look further back in time. To do this we will have to study the recombination history of the universe; i.e how the fraction of ionized matter versus neutral matter has changed during the history of the universe. 
\section{Method} \label{sec:method}
The general expression for optical depth $\tau$ is given by
\begin{equation}
    d\tau\equiv\alpha(s)ds,
\end{equation}
where $\alpha(s)$ is the attenuation coefficient and $s$ is the thickness of the medium. When considering the intergalactic medium the CMB photons have traversed untill we observe them, the main contributor to the attenuation coefficient is thomson scattering off free electrons. The expression for optical depth $\tau$ can be written as
\begin{equation}
    d\tau=n_e\sigma_T ad\eta,
\end{equation}
where $n_e$ is the free electron density, $\sigma_T = \frac{8\pi}{3}\frac{\alpha^2\hbar^2}{m_e^2c^2}$ is the Thomson scattering cross-section,$\alpha$ is the fine structure constant and $a$ is the scale factor. We have now also changed coordinates to the conformal time $\eta$ representing the thickness of the medium. By performing a change of coordinates to $x=log(a)$ we can get a differential equation for $\tau$ given By
\begin{equation}\label{eq:optical_depth_of_x}
    \frac{d\tau}{dx} = -\frac{n_e \sigma_T }{H(x)},
\end{equation}
where $H(x)$ is the Hubble parameter in our new coordinate system. The evolution of the Hubble parameter used in the calculation is described in milestoneI. To solve the ODE for $\tau$ we need to evaluate the electron density $n_e$ as a function of $x=log(a)$. When studying the recombination history of the universe, our model will include some simplifications. We assume the universe mostly consists of hydrogen by setting the helium fraction $Y_p=0$. We also assume that all baryonic matter is represented by the hydrogen nucleus. This makes us able to approximate the proton density $n_H$ as
\begin{equation}
    n_H = n_b \approx \frac{\rho_b}{m_H} = \frac{\Omega_{b,0} \rho_{c,0}}{m_H a^3},
\end{equation}
where $n_b$ is the baryon density, $\rho_{c,0}$ is the critical density of the universe today, $\Omega_{b,0}$ is the density parameter for baryons today and $m_H$ is the hyodrogen mass. A relationship between the quantities $n_e$ and $n_H$ can be expressed through the free electron fraction given by $X_e\equiv\frac{n_e}{n_H}$. The free electron fraction can be solved using two equations dependent on quantities that evolve with $x=log(a)$. The first one is the Saha equation given by
\begin{equation}\label{eq:saha_eq}
    \frac{X_e^2}{1-X_e} = \frac{1}{n_b} \left(\frac{m_eT_b}{2\pi}\right)^{3/2} e^{-\epsilon_0/T_b},
\end{equation}
where $\epsilon_0=13.6$eV is the ionization energy of the hydrogen atom, $m_e$ is the electron mass and $T_b$ is the baryon temperature. The baryon temperature can be approximated by the photon temperature giving $T_b =
T_r = T_{\rm CMB} / a = 2.725 \textrm{K} / a$ \cite{WintherII:2020}. The second equation one can use to solve for the free electron fraction is the Peebles' equation given by
\begin{equation}\label{eq:peeble_eq}
    \frac{dX_e}{dx} = \frac{C_r(T_b)}{H(x)} \left[\beta(T_b)(1-X_e) - n_H\alpha^{(2)}(T_b)X_e^2\right],
\end{equation}
where the reaction rates are given by
\begin{align}
    C_r(T_b) &= \frac{\Lambda_{2s\rightarrow1s} +
    \Lambda_{\alpha}}{\Lambda_{2s\rightarrow1s} + \Lambda_{\alpha} +
    \beta^{(2)}(T_b)}, \\
    \Lambda_{2s\rightarrow1s} &= 8.227 \textrm{s}^{-1}\\
    \Lambda_{\alpha} &= H\frac{(3\epsilon_0)^3}{(8\pi)^2 n_{1s}}\\
    n_{1s} &= (1-X_e)n_H \\
    \beta^{(2)}(T_b) &= \beta(T_b) e^{3\epsilon_0/4T_b} \\
    \beta(T_b) &= \alpha^{(2)}(T_b) \left(\frac{m_e
    T_b}{2\pi}\right)^{3/2} e^{-\epsilon_0/T_b} \\
    \alpha^{(2)}(T_b) &= \frac{64\pi}{\sqrt{27\pi}}
    \frac{\alpha^2}{m_e^2}\sqrt{\frac{\epsilon_0}{T_b}}\phi_2(T_b) \\
    \phi_2(T_b) &= 0.448\ln(\epsilon_0/T_b).
\end{align}
\cite{WintherII:2020}. The Saha equation given by equation \ref{eq:saha_eq} is derived with the assumption that we have the equilibrium reaction $e^- +p\rightleftharpoons H+\gamma$ \cite[p.70]{Dodelson:1282338}. That is the relation between free electrons and protons and neutral hydrogen remains relatively constant. In this case of the early universe all matter is ionized making. Therefore this equation is only valid as long as the free electron fraction $X_e$ is close to one. Therefore we solve we solve the Saha equation when $1\leq X_e<0.99$. This equation can be solved as a regular second order equation where the solution is given by
\begin{equation}
    X_e = \frac{F(x)}{2}\left[-1\pm\sqrt{1+4/F(x)}\right],
\end{equation}
where $F=\frac{1}{n_b} \left(\frac{m_eT_b}{2\pi}\right)^{3/2} e^{-\epsilon_0/T_b}$. Here we only take into account the positive solution as the negative solution has no physical importance. When electrons freezes out and leaves equilibrium we have to resort to the Peebles' equation given by equation \ref{eq:peeble_eq}.
\label{sec:results}

\section{Benchmark}


\bibliography{ref.bib}
\bibliographystyle{aasjournal}
%\begin{thebibliography}{}
%\end{thebibliography}
\end{document}

% End of file `sample62.tex'.
